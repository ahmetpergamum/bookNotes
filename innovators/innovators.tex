\chapter{Innovators}
\label{chp:b2}

Written by Walter Isaacson.

\section{Intro}

In the mid-nineteenth century, Thomas Carlyle declared that ``the history of the
world is but the biography of great men'', and Herbert Spencer responded with a
theory that emphasized the role of societel forces. When it comes to digital-age
innovation, a variety of personal and cultural forces come into play.

The truest creativity of the digital age came from those who were able to
connect the arts and sciences. They believed that beauty mattered. ``I always
thought of myself as a humanities person as a kid, but I liked electronics.''
Jobs told.

The people who were comfortable at this humanities-technology intersection
helped to create the human machine symbiosis that is at the core of this
history.

\section{Foundations}

\subsection{Ada Lovelace (1815-1852)}

Ada realized that math was a lovely language, one that describes the harmonies
of the universe and can be poetic at times. She remained her father's daughter,
with a poetic sensibility that allowed her to view an equation as a brushstroke
that painted an aspect of nature's physical splendor.

Ada asked in 1842 essay ``What is imagination? It is the Combining faculty. It
brings together things, facts, ideas, conceptions, in new, original, endless,
ever-varying combinations\ldots It is that which penetrates into the unseen
worlds around us, the worlds of Science.''

\subsection{Charles Babbage (1791-1871)}

In the 1640s, Blaise Pascal, the French mathematician and philosopher, cretaed a
mechanical calculator to reduce the drudgery of his father's work as a tax
supervisor.

Thirty years later, Gottfriend Liebniz, the German mathematician and
philosopher,tried to improve upon Pascal's contraption with a `stepped reckoner'
that had the capacity to multiply and divide. But he ran into a problem that
would be recurring theme of the digital age. Unlike Pascal, an adroit engineer
who could combine scientific theories with mechanical genius, Leibniz had little
engineering skill and did not surround himself with those who did. So, like many
great theorists who lacked practical collaborators, he was unable to produce
reliably working versions of his device.

Babbage knew of the devices of Pascal and Liebniz, but he was trying to do
something more complex. Babbage had combined innovations that had cropped up in
other fields, a trick of many inventors.

Babbage had already used a metal drum to control how the shafts would turn. But
then he studied the automated loom invented in 1801 by a Frenchmen named
Joseph-Marie Jacquard, which transformed the silk-weaving industry. He invented
a method using cards with holes punched in them to automating the creation of
intricate patterns. Using punch cards rather than drums meant that an unlimited
number of instructions could be input. In addition, the sequence of tasks could
be modified, thus making it easier to devise a general purpose machine that was
reprogrammable.

\subsection{Ada's Notes}

In her ``Notes'' Ada explored four concepts;

\begin{itemize}
\item The first was that of a general-purpose machine, one could not only perform a
preset task but could be reprogrammed  to do a limitless array of tasks.

\item The second was machine's operations did not need to be limited to math and
numbers.

\item Third contribution was to figure out in step-by-step detail workings of what we
now call a computer program or algorithm.

\item Last significant concept that she introduced artificial intelligence: Can
machines think? Ada believed not. A century later this assertion would be dubbed
``Lady Lovelace Objection'' by the computer pioneer Alan Turing.
\end{itemize}

\subsection{Summary}

Babbage's most significant conceptual leap was that such machines did not have
to be set to do only one process, but instead could be programmed and
reprogrammed through the use of punch cards.

Ada saw the beauty and significance of that enchanting notion, and she also
described an even more exciting idea that derived from it: such machines
could process not only numbers but anything that could be notated in symbols.

US Department of Defense named its high-level object-oriented programming
language Ada. However, she has also been ridiculed as delusional, flighty and
only a minor contributor to the ``Notes''.

The reality is that Ada's contribution was both profound and inspirational. More
than Babbage or any other person of her era, she was able to glimpse a future in
which machines would become partners of the human imagination.

\section{The Computer}


